% !TeX root = ../main.tex
\begin{resolution}
论文围绕语言-图像对比学习(CLIP)方法展开预训练与迁移方法研究,选题具有重要的理论意义和应用价值。论文创新性成果如下:
\begin{enumerate}
    \item 提出了一种基于高质量图像分类数据扩展的预训练方法。从对比学习的视角重新设计图像分类任务并引入外部专家知识库增强类别语义信息,有效提升了 CLIP 方法视觉表征的语义对齐效果。
    \item 提出了一种基于特征图自蒸馏增强的视觉任务迁移方法。通过自蒸馏方式,在无需额外数据标注下构建像素级训练目标,有效改善了 CLIP 方法在细粒度视觉任务上的迁移性能。
    \item 提出了一种基于离散扩散模型的语义生成任务迁移方法。针对文本信号特性对离散扩散模型进行改进,实现 CLIP 方法在图像描述生成任务上的迁移,达到了与传统自回归方法相当的性能。
\end{enumerate}

论文内容丰富,写作规范,调研详尽,逻辑清晰,表明作者在本领域具有坚实全面的基础理论和系统深入的专门知识,独立从事科研工作能力强,是一篇优秀的博士学位论文。

答辩过程中,阐述清楚,回答问题正确。经答辩委员会无记名投票表决,一致同意通过论文答辩,并建议授予韦毅轩同学工学博士学位。
  % 论文提出了……

  % 论文取得的主要创新性成果包括:

  % 1. ……

  % 2. ……

  % 3. ……

  % 论文工作表明作者在×××××具有×××××知识,具有××××能力,论文××××,答辩××××。

  % 答辩委员会表决,(×票/一致)同意通过论文答辩,并建议授予×××(姓名)×××(门类)学博士/硕士学位。

\end{resolution}
