% !TeX root = ../main.tex

\begin{acknowledgements}
  时光荏苒,五年博士生涯如白驹过隙。回顾这五年科研时光,是各位老师、朋友、家人的关心与支持一路陪伴我走到现在。在此之际,铭以致谢。

  % 首先,我要衷心感谢我的导师郭百宁老师。郭老师知识渊博、幽默风趣、温和谦逊,有大师风范。
  % 作为微软亚洲研究院的副院长,他熟稔科技的发展和行业的进步,在我的科研项目中提出宝贵的指导意见,也在我的人生选择中为我指出一条明路。
  % 这五年时间里,他为我创造了自由、灵活的良好环境,鼓励我和微软亚洲研究院的各位老师交流学习,也支持我勇敢追求自己兴趣所在。
  % 郭老师的言行品格和谆谆教诲将一直在未来人生道路上警醒我、激励我向远方的未知探索。

  首先,我要衷心感谢我的导师郭百宁老师。郭百宁老师知识渊博、幽默风趣、温和谦逊,有大师风范。
  他熟悉科技的发展和行业的进步,在我的科研项目中提出宝贵的指导意见,也在我的人生选择中为我指出一条明路。
  这五年时间里,他为我创造了自由、灵活的良好环境,鼓励我和微软亚洲研究院的各位老师交流学习,也支持我勇敢追求自己兴趣所在。
  郭百宁老师的言行品格和谆谆教诲将一直在未来人生道路上指引我、激励我向远方的未知探索。
  
  其次,我要衷心感谢在微软亚洲研究院联合培养期间给予我指导的每一位研究员,是他们教会了我科研的方法、梳理问题的逻辑和展示自我的能力。
  感谢曹越博士从最基础的科研方法和科研工具开始,引导我一步一步走上科研的道路,他对问题的深刻认知让我受益匪浅。
  感谢胡瀚博士一直以来的引领和鼓励,在我迷茫的时候指引我、在我受挫的时候激励我,他对问题核心的精准把握值得我终身学习。
  感谢张拯博士和彭厚文博士的言传身教,让我不断提升自我,不断进步。

  我也要感谢在微软亚洲研究院期间合作过和给予我帮助的每一位研究员:童欣博士、Stephen Lin 博士、戴琦博士、王春雨博士、鲍建敏博士、元玉惠博士、陈栋博士、陈鹏博士、董悦博士、刘自成博士、Jianfeng Wang 博士、杨一帆老师、李骥老师。
  也要感谢姚朱亮师兄、杨钰琦师兄、孙春宇师兄、唐彦嵩师兄、徐孟德师兄,祝愿他们事业有成、勇攀高峰。
  还要感谢一起合作过同学,张淼森、李睿航、李晨、宁嘉、耿子刚、尼博林、吴侃、朱子欣、王瑞哲、黄伟泉,祝愿他们学习有成、科研顺利。
  更要感谢解振达、刘泽、林宇桐、胡倞成,与他们的深厚友谊一直支持我,祝他们得偿所愿。

  我的成长也离不开母校的关心和关怀。感谢高等研究院的李丽老师、姜久红老师和王亮老师的支持,也感谢高研博党支部的各位同志的鼓励,感谢支书黄泰榕和支委刘铄、孔舒婷的默默付出,感谢我的室友王颢琛和郑欣阳的陪伴。

  最后,我要感谢我的父母和家人,是他们无私的爱与支持和温暖的怀抱,成为我生命力量的源泉,鼓励我勇攀高峰,祝他们身体健康、心想事成。更要感谢我的爱人单昕霞,在十多年无数个难眠的夜里,给予我坚持的信心和直面自我的勇气。

  二十余载求学路,今日毕、八十余载求知路,方启航。在这个波澜壮阔的时代,衷心地祝愿每一个人都朝着自己的北斗星,扬帆起航。
  
  
\end{acknowledgements}
